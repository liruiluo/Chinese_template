\chapter{高级设置}
本章介绍一些较复杂的设置。

\section{从\CTeX{}宏集继承的功能}
\label{sec:ctex}
chinesedoc 文档类建立在 \CTeX{}宏集的 ctexbook 文档类之上,
因此,ctexbook 文档类所提供的功能均可以使用。

\subsection{字体设置}

在 \texttt{fontset} 选项中,提供了 \texttt{chinesefontauto} 和 \texttt{chinesefontpath} 两种设置方式,
用于定义适合中文文档的四种常用中文字体,包括宋体、黑体、楷体、仿宋。
在字形设计上,还做了针对性的映射调整:
黑体作为宋体的粗体;
楷体作为宋体的斜体;
黑体、楷体、仿宋的假粗体作为其对应字体的粗体形式。

具体字体及对应的 LaTeX 命令如下:
\begin{itemize}
    \item \verb|\songti| —— 宋体,作为默认中文衬线字体,对应命令 \verb|\textrm|;
    \item \verb|\heiti| —— 黑体,作为默认中文粗体字体,与 \verb|\bfseries|、\verb|\textbf|、\verb|\textsf| 命令联动显示粗体;
    \item \verb|\kaishu| —— 楷体,作为默认中文斜体字体,对应命令 \verb|\textit|;
    \item \verb|\fangsong| —— 仿宋,作为默认中文等宽字体,对应命令 \verb|\texttt|。
\end{itemize}

两种选项的区别在于字体的获取方式:
\begin{itemize}
    \item \texttt{chinesefontauto} 会自动从系统中搜索所需字体,适合 Windows 平台,能够免去手动设置字体路径的步骤。
    \item \texttt{chinesefontpath} 则允许用户指定字体文件路径,从当前路径下的 \texttt{chinesefont} 文件夹加载字体,适合缺少字体的平台。
\end{itemize}

具体示例:在 Windows 平台上,应在载入 chinesedoc 文档类时加上 \texttt{fontset=chinesefontauto} 选项。
在其他平台上,则应在载入 chinesedoc 文档类时加上 \texttt{fontset=chinesefontpath} 选项,
并在当前路径新建 \texttt{chinesefont} 文件夹,放置相应的字体文件。

如果想要更换中文字体,可以通过新建 ctex-fontset-myfontset.def 定义自己的 fontset \verb|myfontset|。

在系统装有相应字体时,也可以使用CTEX预定义的六种中文字库:
\begin{itemize}
    \item \verb|adobe|,使用Adobe公司的中文字体,不支持\hologo{pdfLaTeX}。
    \item \verb|fandol|,使用Fandol 中文字体,不支持\hologo{pdfLaTeX}。
    \item \verb|founder|,使用方正公司的中文字体。
    \item \verb|mac|,使用macOS系统下的字体,不支持\hologo{pdfLaTeX}。
    \item \verb|ubuntu|,使用Ubuntu系统下的思源字体,不支持\hologo{pdfLaTeX}。
    \item \verb|windows|,使用 Windows 系统下的中易字体和微软雅黑字体。
\end{itemize}
默认情况下,\CTeX{}宏集根据编译方式和操作系统自动指定相应字库。

\subsection{字号设置}

\texttt{zihao} 的选项只有 -4 | 5 | \texttt{false} 三种,
-4 | 5 将文章默认字号 \texttt{\bfseries\string\normalsize}设置为小四号字或五号字,
\texttt{false}禁用本功能。

\subsection{章节新页模式设置}
文档默认情况下是双面模式,每章都从右页(奇数页)开始。
如果希望改成一章可以从任意页开始(禁止章末空白页),可以在载入 chinesedoc 文档类时加上 \texttt{openany} 选项。

\subsection{论文元素名称设置}
用户可以使用 ctexbook 文档类提供的 \verb|\ctexset| 命令设定文档元素名称:
\begin{Verbatim}
\ctexset{
    appendixname   = {附录},
    bibname        = {参考文献},
    contentsname   = {目录},
    listtablename  = {表格索引},
    listfigurename = {插图索引},
    figurename     = {图},
    tablename      = {表}
}
\end{Verbatim}
例如,将目录的标题改为"目{\quad\quad}录":
\begin{Verbatim}
\ctexset{
    contentsname = {目\quad\quad录}
}
\end{Verbatim}

\section{从其它宏包继承的功能}
\label{sec:thirdparty}

chinesedoc 文档类调用了 geometry、fancyhdr、%
hyperref、graphicx 和 ulem 等几个宏包。
因此,这些宏包所提供的功能均可以使用。

除此之外,chinesedoc 文档类还可能调用以下这些宏包:
\begin{itemize}
    \item 启用 \verb|cnfont| 选项时会调用 amsmath、unicode-math 宏包,不启用
        \verb|cnfont| 选项时会调用 amssymb 宏包。
    \item 启用 \verb|cnfoot| 选项时会调用
        tikz 和 scrextend 宏包。
    \item 启用 \verb|cnspace| 选项时会调用
        tocloft、caption 和 subcaption 宏包。
    \item 启用 \verb|spacing| 选项时会调用 setspace 和
        enumitem 宏包。
\end{itemize}
因此在启用相应选项时,用户可以使用对应宏包所提供的功能。
