\chapter{模板功能}
本章对模板提供的功能和配置项进行介绍说明。

\section{文档类选项}
\label{sec:option}

除非特别说明,否则这一节提到的选项中都是不带"\verb|no|"的版本被启用。

\begin{itemize}
    \item \textbf{\texttt{[no]cnfont}}:
        是否使用中文友好的字体配置,包括中西文字体搭配。

    \item \textbf{\texttt{[no]cnfoot}}:
        是否修改脚注相关格式。
        具体地说,启用 \verb|cnfoot| 选项后会进行以下几项设置:
    \begin{itemize}
        \item 脚注参用带圈的编号。
        \item 页脚中脚注编号使用正文(而非上标)字体。
        \item 页脚中脚注编号和脚注文本之间默认间隔一个空格。
    \end{itemize}

    \item \textbf{\texttt{[no]cnspace}}:
        是否使用中文文档的排版间距设置。
        具体地说,启用 \verb|cnspace| 选项后会修改以下几项设置:
    \begin{itemize}
        \item 正文的行距。
        \item 目录中条目的缩进方式。
        \item 图表标题的字号,以及标题中编号和标题文字之间的间隔方式
    \end{itemize}

    \item \textbf{\texttt{[no]spacing}}:
        是否采用一些常用的调整间距的额外版式设定。
        具体地说,启用 \verb|spacing| 选项后会进行以下几项设置:
    \begin{itemize}
        \item 调用 setspace 宏包以使某些细节处的空间安排更美观。
        \item 采用比 \hologo{LaTeX} 默认设定更加紧密的枚举环境。
        \item 调整枚举环境的缩进,以适应中文排版中的习惯。
    \end{itemize}

    \item \textbf{\texttt{[no]spechap}}:
        是否启用 \verb|\specialchap| 命令用于创建无编号章节。

    \item \textbf{\texttt{[no]pdftoc}}:
        启用 \verb|pdftoc| 选项后,
        用 \verb|\tableofcontents| 命令生成目录时会自动添加"目录"的 pdf 书签。

    \item \textbf{\texttt{[no]pdfprop}}:
        是否自动根据设定的文档信息(如作者、标题等)
        设置生成的 pdf 文档的相应属性。

    \item \textbf{其余文档类选项}:%
        chinesedoc 文档类以 ctexbook 文档类为基础,
        其接受的其余所有文档类选项均被传递给 ctexbook。
        其中可能最常用的选项是 \verb|fontset| 和 \verb|zihao|,
        它们选择中文字体和默认字号。
\end{itemize}

\section{文档信息设定}
\label{sec:info}

这一类命令的语法为
\begin{Verbatim}
  \commandname{具体信息} % commandname 为具体命令的名称。
\end{Verbatim}

这些命令总结如下:
\begin{itemize}
    \item \texttt{\bfseries\string\ctitle}:设定文档中文标题;
    \item \texttt{\bfseries\string\cauthor}:设定作者的中文名;
    \item \texttt{\bfseries\string\date}:设定日期;
    \item \texttt{\bfseries\string\ckeywords}:设定中文关键词;
    \item \texttt{\bfseries\string\etitle}:设定文档英文标题;
    \item \texttt{\bfseries\string\eauthor}:设定作者的英文名;
    \item \texttt{\bfseries\string\ekeywords}:设定英文关键词。
\end{itemize}

例如,如果要设定作者为"张三"("Zhang San"),则可以使用以下命令:
\begin{Verbatim}
\cauthor{张三}
\eauthor{Zhang San}
\end{Verbatim}

\section{摘要}
\label{sec:abstract}

\texttt{\bfseries cabstract} 和 \texttt{\bfseries eabstract} 环境用于编写
中英文摘要。
用户只须要写摘要的正文;标题、作者等部分会自动生成。

\section{目录、表格索引、插图索引}
\label{sec:directory}

目录使用 \texttt{\bfseries \string\tableofcontents} 命令生成。
表格索引使用 \texttt{\bfseries \string\listoftables} 命令生成。
插图索引使用 \texttt{\bfseries \string\listoffigures} 命令生成。

\section{主要符号对照表}
\label{sec:denotation}

参考\verb|chap/Denotation.tex|即可,在\verb|denotation|环境下,使用\verb|\item[X] Y|分别表示符号及其说明。

\section{参考文献}
\label{sec:bibtex}

本模板使用\verb|biblatex-gb7714-2015|宏包进行参考文献管理与格式化,并提供相应的个性化配置方法。

\subsection{参考文献标注方式设置}
如要按 "顺序编码制"标注(\verb|\cite|生成 "[1]、[2]"),使用以下配置:
\begin{Verbatim}
\usepackage[backend=biber,style=gb7714-2015,maxbibnames=3,gbnamefmt=lowercase]{biblatex}
\end{Verbatim}
如要按 "著者-出版年制"标注(\verb|\cite| 生成 "(赵, 2011)",\verb|\citet| 生成 "赵 (2011)"),使用以下配置:
\begin{Verbatim}
\usepackage[backend=biber,style=gb7714-2015ay,maxcitenames=3,
maxbibnames=3,gbnamefmt=lowercase]{biblatex}
\end{Verbatim}

\subsection{正文中的标注格式个性化配置}

可以自定义标注方式下,正文中的标注显示的作者数。
\begin{itemize}
    \item \texttt{\bfseries maxcitenames}:控制正文中的标注最多显示的作者数,若超出会显示为"作者1等"。
    \item \texttt{\bfseries mincitenames}:控制在"等"前显示的作者数量。
    \item 需满足 \texttt{\bfseries maxcitenames} ≥ \texttt{\bfseries mincitenames}。
\end{itemize}
