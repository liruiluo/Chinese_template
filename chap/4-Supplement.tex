\chapter{常见需求实现}
本章介绍一些模板功能之外的常见需求的实现方法。
 
\section{表格}
\label{sec:table}

\subsection{基本三线表}
在学术文档中,通常需要使用三线表,且表格的表序和标题应位于表格的上方,基本用法如表~\ref{tab:example-table-basic}所示。通过导入 \verb|booktabs| 宏包,可以使用 \verb|\toprule|、\verb|\midrule| 和 \verb|\bottomrule| 来控制表格的三条水平线。

\begin{table*}[htb]
    \centering
    \caption{基本三线表。}
    \label{tab:example-table-basic}
    \begin{small}
    \begin{tabular}{@{}lccccc@{}}
        \toprule[1.5pt]
        & \textbf{X} & \textbf{Y} & \textbf{Z} & \textbf{N} & \textbf{M} \\
        \midrule[1pt]
        默认        & 99.99 & 99.99 & 99.99 & 99.99 & 99.99 \\
        \quad w/o X   & 99.99 & 99.99 & 99.99 & 99.99 & 99.99 \\
        \quad w/o Y   & 99.99 & 99.99 & 99.99 & 99.99 & 99.99 \\
        \quad w/o Z   & 99.99 & 99.99 & 99.99 & 99.99 & 99.99 \\
        \quad w/o N   & 99.99 & 99.99 & 99.99 & 99.99 & 99.99 \\
        \quad w/o M   & 99.99 & 99.99 & 99.99 & 99.99 & 99.99 \\
        \bottomrule[1.5pt]
    \end{tabular}
    \end{small}
\end{table*}

\subsection{带脚注的三线表}
如果需要在表格中注明数据来源或添加脚注,可通过在\texttt{minipage}环境中嵌套\texttt{tabular}环境来实现,如表~\ref{tab:example-table-footnote}所示。

\begin{table*}[htb]
    \centering
    \begin{minipage}[t]{0.55\linewidth} %
        \caption{带脚注的三线表。}
        \label{tab:example-table-footnote}
        \begin{small}
        \begin{tabular}{@{}lccccc@{}}
            \toprule[1.5pt]
            & \textbf{X} & \textbf{Y} & \textbf{Z} & \textbf{N} & \textbf{M} \\
            \midrule[1pt]
                默认        & 99.99 & 99.99 & 99.99 & 99.99\footnote{表格中的脚注1} & 99.99 \\
            \quad w/o X   & 99.99 & 99.99 & 99.99 & 99.99 & 99.99 \\
            \quad w/o Y   & 99.99 & 99.99 & 99.99 & 99.99 & 99.99 \\
            \quad w/o Z   & 99.99\footnote{表格中的脚注2} & 99.99 & 99.99 & 99.99 & 99.99 \\
            \quad w/o N   & 99.99 & 99.99 & 99.99 & 99.99 & 99.99 \\
            \quad w/o M   & 99.99 & 99.99 & 99.99 & 99.99 & 99.99 \\
            \bottomrule[1.5pt]
        \end{tabular}
        \end{small}
        \\[6pt] \footnotesize 注:数据来源示例。\\
    \end{minipage}
\end{table*}

\subsection{多级表头效果的三线表}
如果需要实现多级表头效果,可以通过导入 \verb|multirow| 宏包,使用 \verb|\multirow| 和 \verb|\multicolumn| 命令来控制表头的合并效果,如表~\ref{tab:example-table-multi}所示。
\begin{table*}[htbp]
    \centering
    \caption{多级表头效果的三线表。}
    \label{tab:example-table-multi}
    \begin{small}
    \begin{tabular}{@{}l|ccc|ccc@{}}
    \toprule
    \multirow{2}{*}{\textbf{Model}} & \multicolumn{3}{c|}{\textbf{数据集A}} & \multicolumn{3}{c}{\textbf{数据集B}} \\ \cmidrule(l){2-7} 
    & \textbf{指标a}(\%) & \textbf{指标b}(\%) & \textbf{指标c} & \textbf{指标a} (\%) & \textbf{指标b}(\%) & \textbf{指标c} \\ \midrule
        模型A      &99.99  & 99.99  & 99.99  &99.99  & 99.99  & 99.99  \\
        模型B      &99.99  & 99.99  & 99.99  &99.99  & 99.99  & 99.99  \\
    \bottomrule
    \end{tabular}
    \end{small}
\end{table*}

\subsection{续表形式的三线表}
当表格较大,不能在一页内显示时,可以使用 \verb|longtable| 宏包,以"续表"的形式将表格分布到多个页面上,如表~\ref{tab:example-table-continue}所示。

\begin{small}
\begin{longtable}[c]{c*{7}{r}}
    \caption{续表形式的三线表。}
    \label{tab:example-table-continue}\\
    \toprule[1.5pt]
     \multicolumn{1}{c}{年龄} & 性别 & \multicolumn{1}{c}{类型} & \multicolumn{1}{c}{数值1} & \multicolumn{1}{c}{数值2}
    & \multicolumn{1}{c}{数值3} & \multicolumn{1}{c}{数值4} & \multicolumn{1}{c}{数值5} \\
    \multicolumn{1}{c}{(岁)} & & \multicolumn{1}{c}{}&
    \multicolumn{1}{c}{}& \multicolumn{1}{c}{}& \multicolumn{1}{c}{
       }& 状态 & 最大值 \\\midrule[1pt]
    \endfirsthead
    \multicolumn{8}{c}{续表~\thetable\hskip1em 续表示例。}\\
    \toprule[1.5pt]
     \multicolumn{1}{c}{年龄} & 性别 & \multicolumn{1}{c}{类型} & \multicolumn{1}{c}{数值1} & \multicolumn{1}{c}{数值2}
    & \multicolumn{1}{c}{数值3} & \multicolumn{1}{c}{数值4} & \multicolumn{1}{c}{数值5} \\
    \multicolumn{1}{c}{(岁)} & & \multicolumn{1}{c}{}&
    \multicolumn{1}{c}{}& \multicolumn{1}{c}{}& \multicolumn{1}{c}{
       }& 状态 & 最大值 \\\midrule[1pt]
    \endhead
    \hline
    \multicolumn{8}{r}{续下页}
    \endfoot
    \endlastfoot
    25 & 男 & A & 145 & 233 & 1 & 0 & 150 \\
    30 & 女 & B & 130 & 250 & 0 & 1 & 187 \\
    35 & 男 & A & 130 & 204 & 0 & 0 & 172 \\
    40 & 女 & C & 120 & 236 & 0 & 1 & 178 \\
    45 & 男 & B & 120 & 354 & 0 & 1 & 163 \\
    50 & 女 & A & 140 & 192 & 0 & 1 & 148 \\
    55 & 男 & C & 140 & 294 & 0 & 0 & 153 \\
    60 & 女 & B & 120 & 263 & 0 & 1 & 173 \\
    \bottomrule[1.5pt]
\end{longtable}
\footnotesize 注:示例数据。
\end{small}

\section{图片}
\label{sec:figure}

与表格相反,图序和图名需要位于图片的下方。
如果含有子图,每个子图需要具有相应的子图名。

\subsection{插入单个独立的图片}
这里展示如何插入一个简单的图片。实际使用时,请将图片文件放在合适的位置。

\subsection{并排插入多个独立的图片}
如果需要并排插入多个独立的图片,分别编排图序,则可使用\verb|minipage|环境。

\subsection{插入具有多个子图的图片}
如果需要插入具有多个子图的图片,推荐使用\verb|subcaption|宏包。

也可以使用\verb|subfloat|宏包,代码更简短,但是不如\verb|subcaption|宏包灵活。

\section{公式}
\label{sec:equation}

公式部分考虑到写作指南中无关于公式页的说明,并未做改动,使用通用\LaTeX{}规范即可。对于复杂公式需求,可使用\verb|amsmath|宏包结合Mathpix\footnote{\url{https://mathpix.com/}}等自动化识别工具。

\begin{multline*}
\int_a^b\biggl\{\int_a^b[f(x)^2g(y)^2+f(y)^2g(x)^2]
 -2f(x)g(x)f(y)g(y)\,dx\biggr\}\,dy \\
 =\int_a^b\biggl\{g(y)^2\int_a^bf^2+f(y)^2
    \int_a^b g^2-2f(y)g(y)\int_a^b fg\biggr\}\,dy
\end{multline*}

上述公式来源于\citeauthor{liu2003uncertain}的《不确定规划》\cite{liu2003uncertain}。
