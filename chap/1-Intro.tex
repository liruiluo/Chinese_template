\chapter{介绍}
\label{chap:intro}

\def\GitHubLink{\href{https://github.com/example/chinese-template}{GitHub仓库链接(示例)}}

本文档是一个通用的中文LaTeX文档模板,基于优秀的开源项目修改而成,去除了特定学校和机构的信息,保留了优秀的中文排版功能。

本模板参照中文文档排版的通用规范,在保持良好视觉效果的同时,提供了灵活的配置选项和完整的文档结构。
模板重写了功能接口,解决了在不同平台上显示中文字体的问题,
同时提供了一些实用的设置功能,如自定义封面格式、章节样式等。

本模板结构清晰,注释详细,较为易于学习和使用。
希望它能为需要使用 \hologo{LaTeX} 撰写中文文档的用户提供帮助。

\section{关键文件}
\begin{itemize}
    \item \verb|document.tex|:模板的主文件。
    \item \verb|references.bib|:模板的参考文献库。
    \item \verb|chinesedoc.cls|:定义chinesedoc文档类。
    \item \verb|ctex-fontset-chinesefontauto.def|、\verb|ctex-fontset-chinesefontpath.def|:字体配置文件。
    \item \verb|chap/|:各章节内容。
\end{itemize}

\section{编译要求}
本模板仅支持UTF-8文件编码和\hologo{XeLaTeX}编译。
请确保所有 \verb|tex| 文件为UTF-8编码,并使用\hologo{XeLaTeX}编译。

\section{使用方法}

\subsection{环境配置}
环境配置:安装TeX Live,配置好LaTeX Workshop扩展(如果使用VSCode)。

下载模板:获取模板文件后,用合适的LaTeX编辑器打开文件夹。

使用模板:打开 \verb|document.tex|,进行编译即可使用。

\subsection{文档定制}
在 \verb|document.tex| 中修改文档基本信息:
\begin{itemize}
    \item \verb|\ctitle{}|:设置文档标题
    \item \verb|\cauthor{}|:设置作者姓名
    \item \verb|\date{}|:设置日期
    \item \verb|\ckeywords{}|:设置关键词
\end{itemize}

\section{主要特性}

\textbf{功能方面:}

\begin{enumerate}[leftmargin=5em]
    \item 通用的中文文档格式
    \item 字体字号以文档类选项形式设置
    \item 简洁的文档信息设置接口
    \item 支持表格索引、插图索引
    \item 支持主要符号对照表
    \item 规范的脚注样式
    \item 美观的表格和图片格式
    \item 支持子图引用格式
\end{enumerate}

\textbf{格式方面:}

\begin{enumerate}[leftmargin=5em]
    \item 规范的中文排版格式
    \item 美观的章节标题样式
    \item 合理的页面布局和间距
    \item 默认隐藏超链接的显示效果
\end{enumerate}
