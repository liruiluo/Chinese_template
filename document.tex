% 通用中文文档模板
% 基于原北大论文模板修改,去除特定信息,保留中文排版格式

%% 导入 chinesedoc 文档类
\documentclass[fontset=chinesefontpath,zihao=-4,ugly,openany]{chinesedoc}
% 文档类格式控制选项说明:
% [字体设置]
%    - fontset=chinesefontauto  调用系统安装的宋体、黑体、楷体、仿宋字体,适合 Windows 平台
%    - fontset=chinesefontpath  调用项目根目录 chinesefont 文件夹内字体文件,适合缺少字体的平台
% [正文字号]
%    - zihao=-4            设置正文字号为小四号
%    - zihao=5             设置正文字号为五号
% [格式控制]
%    - ugly                启用严格格式,规范排版
%    - english             启用英文格式,模板自动生成的文字会显示为英文
% [章节分页规则]
%    - openany             每章可从任意页开始,避免空白页,适合电子版阅读
%    - openright           每章必须从右页(奇数页)开始,适合打印纸质版

%% 导入 biblatex 宏包
\usepackage[backend=biber,style=gb7714-2015,maxbibnames=3,gbnamefmt=lowercase]{biblatex}
% \usepackage[backend=biber,style=gb7714-2015ay,maxcitenames=3,maxbibnames=3,gbnamefmt=lowercase]{biblatex}
% 宏包格式控制选项说明:
%    - style=gb7714-2015   按 "顺序编码制"标注,示例:\cite 生成 [1]、[2]
%    - style=gb7714-2015ay 按 "著者-出版年制"标注,示例:\cite 生成 (赵, 2011),\citet 生成 赵 (2011)
% [正文中的标注格式]
%    - maxcitenames  控制正文中的标注最多显示的作者数,若超出会显示为"作者1等"
%    - mincitenames  控制在"等"前显示的作者数量
% [参考文献表格式]
%    - maxbibnames   控制参考文献表中,每个条目最多显示的作者数量,超出部分会使用"等"省略
%    - gbnamefmt     控制参考文献表英文姓氏大小写规则

% 设定参考文献列表的字号和行间距
\renewcommand*{\bibfont}{\zihao{5}\linespread{1.27}\selectfont}
% 设定参考文献列表的段间距
\setlength{\bibitemsep}{3bp}
% 载入参考文献数据库(注意不要省略".bib")
\addbibresource{references.bib}

%% 示例文档使用宏包和设定
\usepackage{enumitem,fancyvrb} % 列表相关
\usepackage{booktabs,multirow,longtable,makecell} % 表格相关
\usepackage{hologo} % Tex徽标
\usepackage{pdfpages}
\RecustomVerbatimEnvironment{Verbatim}{Verbatim}{frame=single,tabsize=4,fontsize=\footnotesize}
\renewcommand{\v}[1]{\boldsymbol{#1}}
\newcommand\pkg[1]{\textsf{#1}}
\def\templateversion{v1.0.0}
\def\chinesedoc{chinesedoc \templateversion}

%% 设定文档基本信息
\ctitle{中文文档模板示例}
\cauthor{作者姓名}
\date{二〇二四年十二月}
\ckeywords{中文模板,LaTeX,文档格式}

%% 英文信息(可选)
\etitle{Chinese Document Template Example}
\eauthor{Author Name}
\ekeywords{Chinese Template, LaTeX, Document Format}

%% 设定链接显示效果
\hypersetup{
    hidelinks,                   % 移除链接的字体颜色和边框
    linktoc            = all,    % 目录设置为链接的级别 (none | section | page | all)
    breaklinks         = true,   % 是否允许链接换行
    pdfdisplaydoctitle = true,   % 是否在文件标题属性展示标题而不是文件名
    bookmarksdepth     = 3,      % pdf 书签最大深度
    bookmarksopen      = true,   % pdf 书签是否自动展开
    bookmarksopenlevel = 1       % pdf 书签自动展开级别
}%

\begin{document}
    %% 以下为正文之前的部分,默认不进行章节编号
    \frontmatter
    % 此后到下一 \pagestyle 命令之前不排版页眉或页脚
    \pagestyle{empty}
    % 自动生成封面
    \maketitle
    % 封面要求单面打印,故须新开右页
    \cleardoublepage

    %% 此后到下一 \pagestyle 命令之前正常排版页眉和页脚
    \cleardoublepage
    \pagestyle{plain}
    % 重置页码计数器,用大写罗马数字排版此部分页码
    \setcounter{page}{0}
    \pagenumbering{Roman}
    % 中英文摘要
    \begin{cabstract}
    本文档是一个通用的中文LaTeX文档模板,基于原北京大学论文模板修改而成,去除了特定学校信息和学位论文相关内容,保留了优秀的中文排版格式。
    
    本模板采用XeLaTeX编译器,支持UTF-8编码,提供了良好的中文字体支持和排版效果。模板包含了完整的文档结构,包括封面、摘要、目录、正文章节、参考文献和附录等部分。
    
    使用本模板可以快速创建格式规范、排版美观的中文文档,适用于学术论文、技术报告、项目文档等多种场景。
\end{cabstract}

\begin{eabstract}
    This document is a generic Chinese LaTeX document template, modified from the original Peking University thesis template, with university-specific information and dissertation-related content removed, while preserving excellent Chinese typesetting formats.
    
    This template uses XeLaTeX compiler, supports UTF-8 encoding, and provides good Chinese font support and typesetting effects. The template includes a complete document structure, including cover page, abstract, table of contents, main chapters, references, and appendices.
    
    Using this template, you can quickly create well-formatted and beautifully typeset Chinese documents, suitable for academic papers, technical reports, project documentation, and other scenarios.
\end{eabstract}

    % 自动生成目录
    \tableofcontents
    % 如有需要自动生成表格索引、插图索引
    \listoftables
    \listoffigures
    % 如有需要生成主要符号对照表
    \begin{denotation}

    \item[$x,y,m,n,t$] 标量,通常为变量
    \item[$K,L,D,M,N,T$] 标量,通常为超参数
    \item[$x\in \mathbb{R}^{D}$] D维列向量
    \item[$(x_1,\cdots,x_D)$] D维行向量
    \item[$(x_1,\cdots,x_D)^T$ or $(x_1;\cdots;x_D)^T$]  D维行向量
    \item[$\v A\in \mathbb{R}^{K\times D}$]  大小为$K\times D$的矩阵
    \item[$x\in \mathbb{R}^{KD}$]  ($KD$)维的向量
    \item[$\mathbb{M}_i$ or $\mathbb{M}_i(\v x)$]  第$i$列为$\v 1$(或者$\v x$),其余为$\v 0$的矩阵
    \item[$diag(\v x)$]  对角矩阵,其对角元素为$\v x$
    \item[$\v I_N$ or $I$]  ($N\times N$)的单位阵
    \item[$diag(\v A)$]  列向量,其元素为$\v A$的对角元素
    \item[$\v A \in \mathbb{R}^{D_1\times D_2\times \cdots \times D_K}$]  大小为$D_1\times D_2\times \cdots \times D_K$的张量
    \item[$\{x^{(n)}\}^{N}_{n=1}$]  集合
    \item[$\{(x^{(n)},y^{(n)})\}^{N}_{n=1}$]  数据集
    \item[$\mathcal{N}(\v x;\mu,\sum)$]  变量$x$服从均值为$\mu$,方差为$\sum$的高斯分布

\end{denotation}

\footnotetext[1]{本符号对照表内容选自\citeauthor{qiu2020nndl}老师的《神经网络与深度学习》\cite{qiu2020nndl}一书。}


    %% 以下为正文部分,默认要进行章节编号
    \mainmatter
    \chapter{介绍}
\label{chap:intro}

\def\GitHubLink{\href{https://github.com/example/chinese-template}{GitHub仓库链接(示例)}}

本文档是一个通用的中文LaTeX文档模板,基于优秀的开源项目修改而成,去除了特定学校和机构的信息,保留了优秀的中文排版功能。

本模板参照中文文档排版的通用规范,在保持良好视觉效果的同时,提供了灵活的配置选项和完整的文档结构。
模板重写了功能接口,解决了在不同平台上显示中文字体的问题,
同时提供了一些实用的设置功能,如自定义封面格式、章节样式等。

本模板结构清晰,注释详细,较为易于学习和使用。
希望它能为需要使用 \hologo{LaTeX} 撰写中文文档的用户提供帮助。

\section{关键文件}
\begin{itemize}
    \item \verb|document.tex|:模板的主文件。
    \item \verb|references.bib|:模板的参考文献库。
    \item \verb|chinesedoc.cls|:定义chinesedoc文档类。
    \item \verb|ctex-fontset-chinesefontauto.def|、\verb|ctex-fontset-chinesefontpath.def|:字体配置文件。
    \item \verb|chap/|:各章节内容。
\end{itemize}

\section{编译要求}
本模板仅支持UTF-8文件编码和\hologo{XeLaTeX}编译。
请确保所有 \verb|tex| 文件为UTF-8编码,并使用\hologo{XeLaTeX}编译。

\section{使用方法}

\subsection{环境配置}
环境配置:安装TeX Live,配置好LaTeX Workshop扩展(如果使用VSCode)。

下载模板:获取模板文件后,用合适的LaTeX编辑器打开文件夹。

使用模板:打开 \verb|document.tex|,进行编译即可使用。

\subsection{文档定制}
在 \verb|document.tex| 中修改文档基本信息:
\begin{itemize}
    \item \verb|\ctitle{}|:设置文档标题
    \item \verb|\cauthor{}|:设置作者姓名
    \item \verb|\date{}|:设置日期
    \item \verb|\ckeywords{}|:设置关键词
\end{itemize}

\section{主要特性}

\textbf{功能方面:}

\begin{enumerate}[leftmargin=5em]
    \item 通用的中文文档格式
    \item 字体字号以文档类选项形式设置
    \item 简洁的文档信息设置接口
    \item 支持表格索引、插图索引
    \item 支持主要符号对照表
    \item 规范的脚注样式
    \item 美观的表格和图片格式
    \item 支持子图引用格式
\end{enumerate}

\textbf{格式方面:}

\begin{enumerate}[leftmargin=5em]
    \item 规范的中文排版格式
    \item 美观的章节标题样式
    \item 合理的页面布局和间距
    \item 默认隐藏超链接的显示效果
\end{enumerate}

    \chapter{模板功能}
本章对模板提供的功能和配置项进行介绍说明。

\section{文档类选项}
\label{sec:option}

除非特别说明,否则这一节提到的选项中都是不带"\verb|no|"的版本被启用。

\begin{itemize}
    \item \textbf{\texttt{[no]cnfont}}:
        是否使用中文友好的字体配置,包括中西文字体搭配。

    \item \textbf{\texttt{[no]cnfoot}}:
        是否修改脚注相关格式。
        具体地说,启用 \verb|cnfoot| 选项后会进行以下几项设置:
    \begin{itemize}
        \item 脚注参用带圈的编号。
        \item 页脚中脚注编号使用正文(而非上标)字体。
        \item 页脚中脚注编号和脚注文本之间默认间隔一个空格。
    \end{itemize}

    \item \textbf{\texttt{[no]cnspace}}:
        是否使用中文文档的排版间距设置。
        具体地说,启用 \verb|cnspace| 选项后会修改以下几项设置:
    \begin{itemize}
        \item 正文的行距。
        \item 目录中条目的缩进方式。
        \item 图表标题的字号,以及标题中编号和标题文字之间的间隔方式
    \end{itemize}

    \item \textbf{\texttt{[no]spacing}}:
        是否采用一些常用的调整间距的额外版式设定。
        具体地说,启用 \verb|spacing| 选项后会进行以下几项设置:
    \begin{itemize}
        \item 调用 setspace 宏包以使某些细节处的空间安排更美观。
        \item 采用比 \hologo{LaTeX} 默认设定更加紧密的枚举环境。
        \item 调整枚举环境的缩进,以适应中文排版中的习惯。
    \end{itemize}

    \item \textbf{\texttt{[no]spechap}}:
        是否启用 \verb|\specialchap| 命令用于创建无编号章节。

    \item \textbf{\texttt{[no]pdftoc}}:
        启用 \verb|pdftoc| 选项后,
        用 \verb|\tableofcontents| 命令生成目录时会自动添加"目录"的 pdf 书签。

    \item \textbf{\texttt{[no]pdfprop}}:
        是否自动根据设定的文档信息(如作者、标题等)
        设置生成的 pdf 文档的相应属性。

    \item \textbf{其余文档类选项}:%
        chinesedoc 文档类以 ctexbook 文档类为基础,
        其接受的其余所有文档类选项均被传递给 ctexbook。
        其中可能最常用的选项是 \verb|fontset| 和 \verb|zihao|,
        它们选择中文字体和默认字号。
\end{itemize}

\section{文档信息设定}
\label{sec:info}

这一类命令的语法为
\begin{Verbatim}
  \commandname{具体信息} % commandname 为具体命令的名称。
\end{Verbatim}

这些命令总结如下:
\begin{itemize}
    \item \texttt{\bfseries\string\ctitle}:设定文档中文标题;
    \item \texttt{\bfseries\string\cauthor}:设定作者的中文名;
    \item \texttt{\bfseries\string\date}:设定日期;
    \item \texttt{\bfseries\string\ckeywords}:设定中文关键词;
    \item \texttt{\bfseries\string\etitle}:设定文档英文标题;
    \item \texttt{\bfseries\string\eauthor}:设定作者的英文名;
    \item \texttt{\bfseries\string\ekeywords}:设定英文关键词。
\end{itemize}

例如,如果要设定作者为"张三"("Zhang San"),则可以使用以下命令:
\begin{Verbatim}
\cauthor{张三}
\eauthor{Zhang San}
\end{Verbatim}

\section{摘要}
\label{sec:abstract}

\texttt{\bfseries cabstract} 和 \texttt{\bfseries eabstract} 环境用于编写
中英文摘要。
用户只须要写摘要的正文;标题、作者等部分会自动生成。

\section{目录、表格索引、插图索引}
\label{sec:directory}

目录使用 \texttt{\bfseries \string\tableofcontents} 命令生成。
表格索引使用 \texttt{\bfseries \string\listoftables} 命令生成。
插图索引使用 \texttt{\bfseries \string\listoffigures} 命令生成。

\section{主要符号对照表}
\label{sec:denotation}

参考\verb|chap/Denotation.tex|即可,在\verb|denotation|环境下,使用\verb|\item[X] Y|分别表示符号及其说明。

\section{参考文献}
\label{sec:bibtex}

本模板使用\verb|biblatex-gb7714-2015|宏包进行参考文献管理与格式化,并提供相应的个性化配置方法。

\subsection{参考文献标注方式设置}
如要按 "顺序编码制"标注(\verb|\cite|生成 "[1]、[2]"),使用以下配置:
\begin{Verbatim}
\usepackage[backend=biber,style=gb7714-2015,maxbibnames=3,gbnamefmt=lowercase]{biblatex}
\end{Verbatim}
如要按 "著者-出版年制"标注(\verb|\cite| 生成 "(赵, 2011)",\verb|\citet| 生成 "赵 (2011)"),使用以下配置:
\begin{Verbatim}
\usepackage[backend=biber,style=gb7714-2015ay,maxcitenames=3,
maxbibnames=3,gbnamefmt=lowercase]{biblatex}
\end{Verbatim}

\subsection{正文中的标注格式个性化配置}

可以自定义标注方式下,正文中的标注显示的作者数。
\begin{itemize}
    \item \texttt{\bfseries maxcitenames}:控制正文中的标注最多显示的作者数,若超出会显示为"作者1等"。
    \item \texttt{\bfseries mincitenames}:控制在"等"前显示的作者数量。
    \item 需满足 \texttt{\bfseries maxcitenames} ≥ \texttt{\bfseries mincitenames}。
\end{itemize}

    \chapter{高级设置}
本章介绍一些较复杂的设置。

\section{从\CTeX{}宏集继承的功能}
\label{sec:ctex}
chinesedoc 文档类建立在 \CTeX{}宏集的 ctexbook 文档类之上,
因此,ctexbook 文档类所提供的功能均可以使用。

\subsection{字体设置}

在 \texttt{fontset} 选项中,提供了 \texttt{chinesefontauto} 和 \texttt{chinesefontpath} 两种设置方式,
用于定义适合中文文档的四种常用中文字体,包括宋体、黑体、楷体、仿宋。
在字形设计上,还做了针对性的映射调整:
黑体作为宋体的粗体;
楷体作为宋体的斜体;
黑体、楷体、仿宋的假粗体作为其对应字体的粗体形式。

具体字体及对应的 LaTeX 命令如下:
\begin{itemize}
    \item \verb|\songti| —— 宋体,作为默认中文衬线字体,对应命令 \verb|\textrm|;
    \item \verb|\heiti| —— 黑体,作为默认中文粗体字体,与 \verb|\bfseries|、\verb|\textbf|、\verb|\textsf| 命令联动显示粗体;
    \item \verb|\kaishu| —— 楷体,作为默认中文斜体字体,对应命令 \verb|\textit|;
    \item \verb|\fangsong| —— 仿宋,作为默认中文等宽字体,对应命令 \verb|\texttt|。
\end{itemize}

两种选项的区别在于字体的获取方式:
\begin{itemize}
    \item \texttt{chinesefontauto} 会自动从系统中搜索所需字体,适合 Windows 平台,能够免去手动设置字体路径的步骤。
    \item \texttt{chinesefontpath} 则允许用户指定字体文件路径,从当前路径下的 \texttt{chinesefont} 文件夹加载字体,适合缺少字体的平台。
\end{itemize}

具体示例:在 Windows 平台上,应在载入 chinesedoc 文档类时加上 \texttt{fontset=chinesefontauto} 选项。
在其他平台上,则应在载入 chinesedoc 文档类时加上 \texttt{fontset=chinesefontpath} 选项,
并在当前路径新建 \texttt{chinesefont} 文件夹,放置相应的字体文件。

如果想要更换中文字体,可以通过新建 ctex-fontset-myfontset.def 定义自己的 fontset \verb|myfontset|。

在系统装有相应字体时,也可以使用CTEX预定义的六种中文字库:
\begin{itemize}
    \item \verb|adobe|,使用Adobe公司的中文字体,不支持\hologo{pdfLaTeX}。
    \item \verb|fandol|,使用Fandol 中文字体,不支持\hologo{pdfLaTeX}。
    \item \verb|founder|,使用方正公司的中文字体。
    \item \verb|mac|,使用macOS系统下的字体,不支持\hologo{pdfLaTeX}。
    \item \verb|ubuntu|,使用Ubuntu系统下的思源字体,不支持\hologo{pdfLaTeX}。
    \item \verb|windows|,使用 Windows 系统下的中易字体和微软雅黑字体。
\end{itemize}
默认情况下,\CTeX{}宏集根据编译方式和操作系统自动指定相应字库。

\subsection{字号设置}

\texttt{zihao} 的选项只有 -4 | 5 | \texttt{false} 三种,
-4 | 5 将文章默认字号 \texttt{\bfseries\string\normalsize}设置为小四号字或五号字,
\texttt{false}禁用本功能。

\subsection{章节新页模式设置}
文档默认情况下是双面模式,每章都从右页(奇数页)开始。
如果希望改成一章可以从任意页开始(禁止章末空白页),可以在载入 chinesedoc 文档类时加上 \texttt{openany} 选项。

\subsection{论文元素名称设置}
用户可以使用 ctexbook 文档类提供的 \verb|\ctexset| 命令设定文档元素名称:
\begin{Verbatim}
\ctexset{
    appendixname   = {附录},
    bibname        = {参考文献},
    contentsname   = {目录},
    listtablename  = {表格索引},
    listfigurename = {插图索引},
    figurename     = {图},
    tablename      = {表}
}
\end{Verbatim}
例如,将目录的标题改为"目{\quad\quad}录":
\begin{Verbatim}
\ctexset{
    contentsname = {目\quad\quad录}
}
\end{Verbatim}

\section{从其它宏包继承的功能}
\label{sec:thirdparty}

chinesedoc 文档类调用了 geometry、fancyhdr、%
hyperref、graphicx 和 ulem 等几个宏包。
因此,这些宏包所提供的功能均可以使用。

除此之外,chinesedoc 文档类还可能调用以下这些宏包:
\begin{itemize}
    \item 启用 \verb|cnfont| 选项时会调用 amsmath、unicode-math 宏包,不启用
        \verb|cnfont| 选项时会调用 amssymb 宏包。
    \item 启用 \verb|cnfoot| 选项时会调用
        tikz 和 scrextend 宏包。
    \item 启用 \verb|cnspace| 选项时会调用
        tocloft、caption 和 subcaption 宏包。
    \item 启用 \verb|spacing| 选项时会调用 setspace 和
        enumitem 宏包。
\end{itemize}
因此在启用相应选项时,用户可以使用对应宏包所提供的功能。

    \chapter{常见需求实现}
本章介绍一些模板功能之外的常见需求的实现方法。
 
\section{表格}
\label{sec:table}

\subsection{基本三线表}
在学术文档中,通常需要使用三线表,且表格的表序和标题应位于表格的上方,基本用法如表~\ref{tab:example-table-basic}所示。通过导入 \verb|booktabs| 宏包,可以使用 \verb|\toprule|、\verb|\midrule| 和 \verb|\bottomrule| 来控制表格的三条水平线。

\begin{table*}[htb]
    \centering
    \caption{基本三线表。}
    \label{tab:example-table-basic}
    \begin{small}
    \begin{tabular}{@{}lccccc@{}}
        \toprule[1.5pt]
        & \textbf{X} & \textbf{Y} & \textbf{Z} & \textbf{N} & \textbf{M} \\
        \midrule[1pt]
        默认        & 99.99 & 99.99 & 99.99 & 99.99 & 99.99 \\
        \quad w/o X   & 99.99 & 99.99 & 99.99 & 99.99 & 99.99 \\
        \quad w/o Y   & 99.99 & 99.99 & 99.99 & 99.99 & 99.99 \\
        \quad w/o Z   & 99.99 & 99.99 & 99.99 & 99.99 & 99.99 \\
        \quad w/o N   & 99.99 & 99.99 & 99.99 & 99.99 & 99.99 \\
        \quad w/o M   & 99.99 & 99.99 & 99.99 & 99.99 & 99.99 \\
        \bottomrule[1.5pt]
    \end{tabular}
    \end{small}
\end{table*}

\subsection{带脚注的三线表}
如果需要在表格中注明数据来源或添加脚注,可通过在\texttt{minipage}环境中嵌套\texttt{tabular}环境来实现,如表~\ref{tab:example-table-footnote}所示。

\begin{table*}[htb]
    \centering
    \begin{minipage}[t]{0.55\linewidth} %
        \caption{带脚注的三线表。}
        \label{tab:example-table-footnote}
        \begin{small}
        \begin{tabular}{@{}lccccc@{}}
            \toprule[1.5pt]
            & \textbf{X} & \textbf{Y} & \textbf{Z} & \textbf{N} & \textbf{M} \\
            \midrule[1pt]
                默认        & 99.99 & 99.99 & 99.99 & 99.99\footnote{表格中的脚注1} & 99.99 \\
            \quad w/o X   & 99.99 & 99.99 & 99.99 & 99.99 & 99.99 \\
            \quad w/o Y   & 99.99 & 99.99 & 99.99 & 99.99 & 99.99 \\
            \quad w/o Z   & 99.99\footnote{表格中的脚注2} & 99.99 & 99.99 & 99.99 & 99.99 \\
            \quad w/o N   & 99.99 & 99.99 & 99.99 & 99.99 & 99.99 \\
            \quad w/o M   & 99.99 & 99.99 & 99.99 & 99.99 & 99.99 \\
            \bottomrule[1.5pt]
        \end{tabular}
        \end{small}
        \\[6pt] \footnotesize 注:数据来源示例。\\
    \end{minipage}
\end{table*}

\subsection{多级表头效果的三线表}
如果需要实现多级表头效果,可以通过导入 \verb|multirow| 宏包,使用 \verb|\multirow| 和 \verb|\multicolumn| 命令来控制表头的合并效果,如表~\ref{tab:example-table-multi}所示。
\begin{table*}[htbp]
    \centering
    \caption{多级表头效果的三线表。}
    \label{tab:example-table-multi}
    \begin{small}
    \begin{tabular}{@{}l|ccc|ccc@{}}
    \toprule
    \multirow{2}{*}{\textbf{Model}} & \multicolumn{3}{c|}{\textbf{数据集A}} & \multicolumn{3}{c}{\textbf{数据集B}} \\ \cmidrule(l){2-7} 
    & \textbf{指标a}(\%) & \textbf{指标b}(\%) & \textbf{指标c} & \textbf{指标a} (\%) & \textbf{指标b}(\%) & \textbf{指标c} \\ \midrule
        模型A      &99.99  & 99.99  & 99.99  &99.99  & 99.99  & 99.99  \\
        模型B      &99.99  & 99.99  & 99.99  &99.99  & 99.99  & 99.99  \\
    \bottomrule
    \end{tabular}
    \end{small}
\end{table*}

\subsection{续表形式的三线表}
当表格较大,不能在一页内显示时,可以使用 \verb|longtable| 宏包,以"续表"的形式将表格分布到多个页面上,如表~\ref{tab:example-table-continue}所示。

\begin{small}
\begin{longtable}[c]{c*{7}{r}}
    \caption{续表形式的三线表。}
    \label{tab:example-table-continue}\\
    \toprule[1.5pt]
     \multicolumn{1}{c}{年龄} & 性别 & \multicolumn{1}{c}{类型} & \multicolumn{1}{c}{数值1} & \multicolumn{1}{c}{数值2}
    & \multicolumn{1}{c}{数值3} & \multicolumn{1}{c}{数值4} & \multicolumn{1}{c}{数值5} \\
    \multicolumn{1}{c}{(岁)} & & \multicolumn{1}{c}{}&
    \multicolumn{1}{c}{}& \multicolumn{1}{c}{}& \multicolumn{1}{c}{
       }& 状态 & 最大值 \\\midrule[1pt]
    \endfirsthead
    \multicolumn{8}{c}{续表~\thetable\hskip1em 续表示例。}\\
    \toprule[1.5pt]
     \multicolumn{1}{c}{年龄} & 性别 & \multicolumn{1}{c}{类型} & \multicolumn{1}{c}{数值1} & \multicolumn{1}{c}{数值2}
    & \multicolumn{1}{c}{数值3} & \multicolumn{1}{c}{数值4} & \multicolumn{1}{c}{数值5} \\
    \multicolumn{1}{c}{(岁)} & & \multicolumn{1}{c}{}&
    \multicolumn{1}{c}{}& \multicolumn{1}{c}{}& \multicolumn{1}{c}{
       }& 状态 & 最大值 \\\midrule[1pt]
    \endhead
    \hline
    \multicolumn{8}{r}{续下页}
    \endfoot
    \endlastfoot
    25 & 男 & A & 145 & 233 & 1 & 0 & 150 \\
    30 & 女 & B & 130 & 250 & 0 & 1 & 187 \\
    35 & 男 & A & 130 & 204 & 0 & 0 & 172 \\
    40 & 女 & C & 120 & 236 & 0 & 1 & 178 \\
    45 & 男 & B & 120 & 354 & 0 & 1 & 163 \\
    50 & 女 & A & 140 & 192 & 0 & 1 & 148 \\
    55 & 男 & C & 140 & 294 & 0 & 0 & 153 \\
    60 & 女 & B & 120 & 263 & 0 & 1 & 173 \\
    \bottomrule[1.5pt]
\end{longtable}
\footnotesize 注:示例数据。
\end{small}

\section{图片}
\label{sec:figure}

与表格相反,图序和图名需要位于图片的下方。
如果含有子图,每个子图需要具有相应的子图名。

\subsection{插入单个独立的图片}
这里展示如何插入一个简单的图片。实际使用时,请将图片文件放在合适的位置。

\subsection{并排插入多个独立的图片}
如果需要并排插入多个独立的图片,分别编排图序,则可使用\verb|minipage|环境。

\subsection{插入具有多个子图的图片}
如果需要插入具有多个子图的图片,推荐使用\verb|subcaption|宏包。

也可以使用\verb|subfloat|宏包,代码更简短,但是不如\verb|subcaption|宏包灵活。

\section{公式}
\label{sec:equation}

公式部分考虑到写作指南中无关于公式页的说明,并未做改动,使用通用\LaTeX{}规范即可。对于复杂公式需求,可使用\verb|amsmath|宏包结合Mathpix\footnote{\url{https://mathpix.com/}}等自动化识别工具。

\begin{multline*}
\int_a^b\biggl\{\int_a^b[f(x)^2g(y)^2+f(y)^2g(x)^2]
 -2f(x)g(x)f(y)g(y)\,dx\biggr\}\,dy \\
 =\int_a^b\biggl\{g(y)^2\int_a^bf^2+f(y)^2
    \int_a^b g^2-2f(y)g(y)\int_a^b fg\biggr\}\,dy
\end{multline*}

上述公式来源于\citeauthor{liu2003uncertain}的《不确定规划》\cite{liu2003uncertain}。


    %% 正文中的附录部分
    \appendix
    % 要使参考文献列表参与章节编号,可将"bibintoc"改为"bibnumbered"
    \printbibliography[heading=bibintoc]
    \chapter{附录示例}


    %% 以下为正文之后的部分,默认不进行章节编号
    \backmatter
    \acks
在此文档的创建过程中,感谢开源社区提供的优秀LaTeX模板基础。
感谢各位开发者的辛勤贡献,为中文文档排版提供了便利。
本模板在原有基础上进行了修改和优化,希望能为更多用户提供帮助。

感谢所有使用本模板的用户,您的反馈和建议是我们不断改进的动力。


\end{document} 